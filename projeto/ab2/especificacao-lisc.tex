
\documentclass[
  % -- opções da classe memoir --
  12pt,				% tamanho da fonte
  % openright,			% capítulos começam em pág ímpar (insere página vazia caso preciso)
  oneside,			% para impressão somente frente. Oposto a twoside (frente e verso)
  a4paper,			% tamanho do papel. 
  % -- opções da classe abntex2 --
  %chapter=TITLE,		% títulos de capítulos convertidos em letras maiúsculas
  %section=TITLE,		% títulos de seções convertidos em letras maiúsculas
  %subsection=TITLE,	% títulos de subseções convertidos em letras maiúsculas
  %subsubsection=TITLE,% títulos de subsubseções convertidos em letras maiúsculas
  % -- opções do pacote babel --
  english,			% idioma adicional para hifenização
  french,				% idioma adicional para hifenização
  spanish,			% idioma adicional para hifenização
  brazil,				% o último idioma é o principal do documento
]{abntex2}


% ---
% PACOTES
% ---

% ---
% Pacotes fundamentais 
% ---
\usepackage{cmap}				% Mapear caracteres especiais no PDF
\usepackage{lmodern}			% Usa a fonte Latin Modern
\usepackage[T1]{fontenc}		% Selecao de codigos de fonte.
\usepackage[utf8]{inputenc}		% Codificacao do documento (conversão automática dos acentos)
\usepackage{indentfirst}		% Indenta o primeiro parágrafo de cada seção.
\usepackage{color}				% Controle das cores
\usepackage{graphicx}			% Inclusão de gráficos
\usepackage{float}

% ---

% ---
% Pacotes adicionais, usados no anexo do modelo de folha de identificação
% ---
\usepackage{multicol}
\usepackage{array}
\usepackage{tabu}
\usepackage{multirow}
\usepackage{listings}
\usepackage{hyperref}
\definecolor{codegreen}{rgb}{0,0.6,0}
\definecolor{codegray}{rgb}{0.5,0.5,0.5}
\definecolor{codepurple}{rgb}{0.58,0,0.82}
\definecolor{backcolour}{rgb}{0.95,0.95,0.92}
\definecolor{codered}{rgb}{0.960,0,0.054}
\definecolor{codeblue}{rgb}{0, 0.141, 0.960}
\lstdefinestyle{mystyle}{
    backgroundcolor=\color{backcolour},   
    commentstyle=\color{codegreen},
    keywordstyle=\color{codeblue},
    numberstyle=\tiny\color{codepurple},
    stringstyle=\color{codemagenta},
    basicstyle=\footnotesize,
    breakatwhitespace=false,         
    breaklines=true,                 
    captionpos=b,                    
    keepspaces=true,                 
    numbers=left,                    
    numbersep=5pt,                  
    showspaces=false,                
    alsoletter={'},
    showstringspaces=false,
    showtabs=false,                  
    tabsize=4
}

\lstset{style=mystyle,frame=tb, morekeywords={defun, int, defint, echo, return, deffloat, defbool,
    defchar, if, else, set, string, defstring, read, while, for,
    break, continue, nil}}

\renewcommand{\lstlistingname}{Código}% Listing -> Algorithm
% ---
	
% ---
% Pacotes adicionais, usados apenas no âmbito do Modelo Canônico do abnteX2
% ---
\usepackage{lipsum}				% para geração de dummy text
% ---

% ---
% Pacotes de citações
% ---
\usepackage[brazilian,hyperpageref]{backref}	 % Paginas com as citações na bibl
\usepackage[alf]{abntex2cite}	% Citações padrão ABNT
\usepackage{tabularx}


% --- 
% CONFIGURAÇÕES DE PACOTES
% --- 

% ---
% Configurações do pacote backref
% Usado sem a opção hyperpageref de backref
\renewcommand{\backrefpagesname}{Citado na(s) página(s):~}
% Texto padrão antes do número das páginas
\renewcommand{\backref}{}
% Define os textos da citação
\renewcommand*{\backrefalt}[4]{
  \ifcase #1 %
  Nenhuma citação no texto.%
  \or
  Citado na página #2.%
  \else
  Citado #1 vezes nas páginas #2.%
  \fi}%
% ---

% ---
% Informações de dados para CAPA e FOLHA DE ROSTO
% ---
\titulo{Especificação da gramática da linguagem de programação LisC}
\autor{Lucas Peixoto de Almeida Cavalcante}
\local{Brasil}
\data{Maceió/Al\\2019.1}
\instituicao{%
  Universidade Federal de Alagoas
  \par
  Instituto de Computação
}
\tipotrabalho{Relatório técnico}
\preambulo{Especificação da gramática da linguagem de programação LisC, definida
  pelo aluno, para a disciplina de Compiladores, correspondente à
  parte da avaliação da AB2 do semestre de 2019.1, sob orientação do
  \textbf{Prof. Alcino Dall Igna Jr.}}
% ---

% ---
% Configurações de aparência do PDF final

% alterando o aspecto da cor azul
\definecolor{blue}{RGB}{41,5,195}

% informações do PDF
\makeatletter
\hypersetup{
  %pagebackref=true,
  pdftitle={\@title}, 
  pdfauthor={\@author},
  pdfsubject={\imprimirpreambulo},
  pdfcreator={LaTeX with abnTeX2},
  pdfkeywords={abnt}{latex}{abntex}{abntex2}{relatório técnico}, 
  colorlinks=true,       		% false: boxed links; true: colored links
  linkcolor=blue,          	% color of internal links
  citecolor=blue,        		% color of links to bibliography
  filecolor=magenta,      		% color of file links
  urlcolor=blue,
  bookmarksdepth=4
}
\makeatother
% --- 

% --- 
% Espaçamentos entre linhas e parágrafos 
% --- 

% O tamanho do parágrafo é dado por:
\setlength{\parindent}{1.3cm}

% Controle do espaçamento entre um parágrafo e outro:
\setlength{\parskip}{0.2cm}  % tente também \onelineskip

% ---
% compila o indice
% ---
\makeindex
% ---

% ----
% Início do documento
% ----
\begin{document}

% Retira espaço extra obsoleto entre as frases.
\frenchspacing 

% ----------------------------------------------------------
% ELEMENTOS PRÉ-TEXTUAIS
% ----------------------------------------------------------
% \pretextual

% ---
% Capa
% ---
%% \imprimircapa
% ---

% ---
% Folha de rosto
% (o * indica que haverá a ficha bibliográfica)
% ---
\imprimirfolhaderosto*
% ---

% ---
% inserir o sumario
% ---

\tableofcontents*

% ----------------------------------------------------------
% ELEMENTOS TEXTUAIS  (necessário para incluir número nas páginas)
% ----------------------------------------------------------
\textual
\chapter{Introdução}
\label{cha:intro}
Esse documento descreve a gramática construída da linguagem de
programação LisC, que deverá ser respeitada na implementação do
analisador sintático em C. O analisador sintático escolhido é o
preditivo recursivo. Diante da escolha deste analisador sintático,
teremos também nesse documento as alterações necessárias na gramática
para que a gramática se torne uma LL(1), pré-requisito obrigatório de
todo analisador sintático preditivo recursivo.


\chapter{Gramática Original}
\label{cha:gramatica-original}
O \textbf{código} \ref{code:gramatica-original} apresenta a gramática
original da linguagem.

\begin{lstlisting}[label={code:gramatica-original},caption={Gramática
    original da linguagem de programação LisC.},style={mystyle}]
Programa = ListaDecl

ListaDecl = ListaDecl Decl
ListaDecl = Decl

Decl = DeclVariaveis
Decl = DeclFuncao

DeclVariaveis = DefTipo ListaVariaveis ';'

DefTipo = 'defint'
DefTipo = 'defbool'
DefTipo = 'defchar'
DefTipo = 'defstring'
DefTipo = 'deffloat'

ListaVariaveis = ListaVariaveis ',' Variavel
ListaVariaveis = Variavel

Variavel = 'id'
Variavel = 'id' ExpConcat

DeclFuncao = 'defun' 'id' '(' ListaParam ')' '(' TipoReturn ') '{' ListaSent '}'

ListaParam = ListaParam ',' Param
ListaParam = Param

Param = Tipo 'id'

TipoReturn = 'void'
TipoReturn = Tipo

ListaSent = ListaSent Sent
ListaSent = Sent

Sent = DeclVariaveis
Sent = Comando

Comando = Continue
Comando = Break
Comando = Atribuicao
Comando = ChamadaFunc
Comando = Return
Comando = Read
Comando = Echo
Comando = If
Comando = For
Comando = While

Continue = 'continue' ';'

Break = 'break' ';'

Atribuicao = 'set' 'id' ExpConcat ';'

ChamadaFunc = 'id' '(' ListaArg ')' ';'

Return = 'return' ExpConcat ';'

Read = 'read' '(' ListaArg ')' ';'

Echo = 'echo' '(' ExpConcat ')'
Echo = 'echo' '(' ExpConcat ')' '(' ListaArg ')'

If = 'if' '(' ExpBooleana ')' '{' ListaSent '}' Else

Else = 'else' '{' ListaSent '}'
Else = epsilon

For = 'for' '(' 'id' ',' ExpAritmetica ',' ExpAritmetica ForPasso ')' '{' ListaSent '}'

While = 'while' '(' ExpBooleana ')' '{' ListaSent '}'

ForPasso = ',' ExpAritmetica
ForPasso = epsilon

ListaArg = ListaArg ',' 'id'
ListaArg = 'id'

Tipo = 'int'
Tipo = 'char'
Tipo = 'string'
Tipo = 'float'
Tipo = 'bool'

ExpAritmetica = ExpAritmetica 'op_adi' TermAritmetico
ExpAritmetica = TermAritmetico

ExpBooleana = ExpBooleana 'op_logic' TermBooleano
ExpBooleana = TermBooleano

ExpConcat = ExpConcat '<<' ExpAritmetica
ExpConcat = ExpBooleana

TermBooleano = '!' TermBooleano
TermBooleano = ExpAritmetica 'op_relacional' ExpAritmetica
TermBooleano = ExpAritmetica

TermAritmetico = TermAritmetico 'op_mult_div' TermFormatacao
TermAritmetico = TermFormatacao

TermFormatacao = TermFormatacao 'op_format' FatorAritmetico
TermFormatacao = FatorAritmetico

FatorAritmetico = '-' FatorAritmetico
FatorAritmetico = 'id'
FatorAritmetico = 'const_float'
FatorAritmetico = 'const_char'
FatorAritmetico = 'const_bool'
FatorAritmetico = 'const_int'
FatorAritmetico = 'const_string'
FatorAritmetico = '(' ExpBooleana ')'

\end{lstlisting}

\chapter{Gramática LL(1)}
\label{cha:gramatica-ll1}

\begin{lstlisting}[label={code:gramatica-original},caption={Gramática
    original da linguagem de programação LisC.},style={mystyle}]
Programa = ListaDecl

ListaDecl = Decl ListaDeclR

ListaDeclR = Decl ListaDeclR
ListaDeclR = epsilon

Decl = DeclVariaveis
Decl = DeclFuncao

DeclVariaveis = DefTipo ListaVariaveis ';'

DefTipo = 'defint'
DefTipo = 'defbool'
DefTipo = 'defchar'
DefTipo = 'defstring'
DefTipo = 'deffloat'

ListaVariaveis = Variavel ListaVariaveisR

ListaVariaveisR = ',' Variavel ListaVariaveisR
ListaVariaveisR = epsilon

Variavel = 'id' VariavelF

VariavelF = ExpConcat
VariavelF = epsilon

DeclFuncao = 'defun' 'id' '(' ListaParam ')' '(' TipoReturn ') '{' ListaSent '}'

ListaParam = Param ListaParamR

ListaParamR = ',' Param ListaParamR
ListaParamR = epsilon

Param = Tipo 'id'

TipoReturn = 'void'
TipoReturn = Tipo

ListaSent = Sent ListaSentR

ListaSentR = Sent ListaSentR
ListaSentR = epsilon

Sent = DeclVariaveis
Sent = Comando

Comando = Continue
Comando = Break
Comando = Atribuicao
Comando = ChamadaFunc
Comando = Return
Comando = Read
Comando = Echo
Comando = If
Comando = For
Comando = While

Continue = 'continue' ';'

Break = 'break' ';'

Atribuicao = 'set' 'id' ExpConcat ';'

ChamadaFunc = 'id' '(' ListaArg ')' ';'

Return = 'return' ExpConcat ';'

Read = 'read' '(' ListaArg ')' ';'

Echo = 'echo' '(' ExpConcat ')' EchoF

EchoF = '(' ListaArg ')' ';'
EchoF = ';'

If = 'if' '(' ExpBooleana ')' '{' ListaSent '}' Else

Else = 'else' '{' ListaSent '}'
Else = epsilon

For = 'for' '(' 'id' ',' ExpAritmetica ',' ExpAritmetica ForPasso ')' '{' ListaSent '}'

While = 'while' '(' ExpBooleana ')' '{' ListaSent '}'

ForPasso = ',' ExpAritmetica
ForPasso = epsilon

ListaArg = 'id' ListaArgR

ListaArgR = ',' 'id' ListaArgR
ListaArgR = epsilon

Tipo = 'int'
Tipo = 'char'
Tipo = 'string'
Tipo = 'float'
Tipo = 'bool'

ExpAritmetica = TermAritmetico ExpAritmeticaR

ExpAritmeticaR = 'op_adi' TermAritmetico ExpAritmeticaR
ExpAritmeticaR = epsilon

ExpBooleana = TermBooleano ExpBooleanaR

ExpBooleanaR = 'op_logic' TermBooleano ExpBooleanaR
ExpBooleanaR = epsilon

ExpConcat = ExpBooleana ExpConcatR

ExpConcatR = '<<' ExpAritmetica ExpConcatR
ExpConcatR = epsilon

TermBooleano = '!' TermBooleano
TermBooleano = ExpAritmetica TermBooleanoF

TermBooleanoF = 'op_relacional' ExpAritmetica
TermBooleanoF = epsilon

TermAritmetico = TermFormatacao TermAritmeticoR

TermAritmeticoR = 'op_mult_div' TermFormatacao TermFormatacaoR
TermAritmeticoR = epsilon

TermFormatacao = FatorAritmetico TermFormatacaoR

TermFormatacaoR = 'op_format' FatorAritmetico TermFormatacaoR
TermFormatacaoR = epsilon

FatorAritmetico = '-' FatorAritmetico
FatorAritmetico = 'id'
FatorAritmetico = 'const_float'
FatorAritmetico = 'const_char'
FatorAritmetico = 'const_bool'
FatorAritmetico = 'const_int'
FatorAritmetico = 'const_string'
FatorAritmetico = '(' ExpBooleana ')'

\end{lstlisting}




% ----------------------------------------------------------
% ELEMENTOS PÓS-TEXTUAIS
% ----------------------------------------------------------
\postextual

%---------------------------------------------------------------------
% INDICE REMISSIVO
%---------------------------------------------------------------------

\printindex

\end{document}

%%% Local Variables:
%%% mode: latex
%%% TeX-master: t
%%% End:
