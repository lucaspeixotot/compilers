% Created 2019-08-29 qui 21:00
% Intended LaTeX compiler: pdflatex
\documentclass[11pt]{article}
\usepackage[utf8]{inputenc}
\usepackage[T1]{fontenc}
\usepackage{graphicx}
\usepackage{grffile}
\usepackage{longtable}
\usepackage{wrapfig}
\usepackage{rotating}
\usepackage[normalem]{ulem}
\usepackage{amsmath}
\usepackage{textcomp}
\usepackage{amssymb}
\usepackage{capt-of}
\usepackage{hyperref}
\author{Lucas Peixoto}
\date{\today}
\title{}
\hypersetup{
 pdfauthor={Lucas Peixoto},
 pdftitle={},
 pdfkeywords={},
 pdfsubject={},
 pdfcreator={Emacs 26.2 (Org mode 9.1.9)}, 
 pdflang={English}}
\begin{document}

\tableofcontents

\section{Tradução dirigida por sintaxe (Capítulo 4)}
\label{sec:org9d6c23c}
Tradução dirigida por sintaxe é uma técnica que permite realizar
tradução (geração de código) concomitantemente com a análise
sintática via ações semânticas que são associadas as regras de
produção da gramática. A execução dessas ações pode gerar ou interpretar código,
armazenar informações na tabela de símbolos, emitir mensages de erro
e etc.

Os símbolos gramaticais passarão a conter atributos capazes de
armazenar valores durante o processo de reconhecimento. 

\subsection{Esquemas de tradução}
\label{sec:org7e389a7}
Um esquema de tradução é uma extensão de uma gramática livre de
contexto, extensão essa realizada através da associação de
atributos aos símbolos gramaticais e de ações semânticas as regras
de produção. Um atributo de um símbolo pode conter um valor
numérico, uma cadeia de caracteres, um tipo de dado, um endereço de
memória e etc. Ações semânticas podem ser avaliações de atributos
ou chamadas a procedimentos e funções. Com exceção dos atributos
dos terminais que são calculados pelo analisador léxico, os valores
dos demais atributos e variáveis são calculados durante a execução
das ações semânticas.

Atributos podem ser sintetizados ou herdados. Dado uma sentença de
entrada, a ela irá corresponder uma árvore de derivação, a qual
será construída durante o processo de análise. A execução das açõe
ssemânticas irá definir os valores dos atributos dos nós da árvore
para essa sentença de entrada. Deve ser observado que todas as
informações referentes a uma sentança estão contidas na sentença
original. Isto é, na sequencia de tokens da cadeia de entrada. Como
os tokens ficam nas folhas da árvore de derivação, isso significa
que todas as informações para um analisador estão originalmente
nessas folhas.

As regras semânticas computam valores de atributos a partir de
outros atributos e isso estabelece uma dependência entre as regras,
pois um atributo só pode ser calculado depois que todos os
atributos dos quais ele depende tiverem sido calculados. Portanto,
a ordem em que as regras de produção são usadas no processo de
reconhecimento pode não ser necessariamente a mesma ordem em que as
açõe ssemânticas correspondentes são executadas. A ordenação
correta para realizar os cálculos dos atributos é representada
através de um grafo dirigido, chamado de grafo de dependência. Esse
grafo determina uma sequência de avaliação (ou escalonamento) para
as regras semânticas.

Os esquemas S-atribuídos e os esquemas L-atribuídos possuem a
característica de executar os esquemas de traduções e suas ações
semânticas num único passo. A árvore de derivação que mostra os
valores dos atributos associados a cada nó é chamada de árvore de
derivação anotada.

Mesmo que o cálculo dos atributos de um símbolo possa variar em
fução da localização do símbolo na árvore, cada instanciaçao desse
símbolo terá sempre a mesma configuração (mesmo conjunto de
atributos). Isto é, o número de atributos e os seus tipos não
variam em função do ponto em que o símbolo aparece na árvore de
derivação.

\subsection{Gramática de atributos}
\label{sec:org31d72c2}
As ações semânticas podem produzir efeitos colaterais tais como
imprimir um valor, armazenar um literal em uma tabela ou em um
arquivo, atualizar uma variável global, etc. Quando o esquema de
tradução não produz efeitos colaterais, ele é dito ser uma
gramática de atributos. Nesse caso, as ações semânticas são
atribuições ou funções envolvendo unicamente os atributos do
esquema (não são usadas variáveis globais, arquivos, etc).

\section{Geração de código intermediário (Capítulo 5)}
\label{sec:org882e870}
A geração de código intermediário é a transformação da árvore de
derivação em um segmento de código. Esse código pode, eventualmente,
ser o código objeto final, mas, na maioria dos casos constitui-se
num código intermediário, pois a tradução de código fonte para
objeto em mais de um passo apresenta algumas vantagens, como:

\begin{itemize}
\item Possibilita a otimização do código intermediário, de modo a
obter-se o código objeto final mais eficiente;
\item simplifica a implementação do compilador, resolvendo,
gradativamente, as dificuldades da passagem de código fonte para
objeto, já que o código fonte pode ser visto como um texto
condensado que "explode" em inúmeras instruções de baixo-nivel;
\item possibilita a tradução de código intermediário para diversas
máquinas.
\end{itemize}

A desvantagem de gerar código intermediário é que o computador
requer um passo a mais. A tradução direta do código fonte para o
objeto leva a uma compilação mais rápida.

A única diferença entre o código intermediário e o código objeto
final é que o código intermediário não especifica detalhes da
máquina alvo.

\subsection{Linguagens intermediárias}
\label{sec:org0b24f49}
Os diferentes tipos de linguagem intermediária fazem parte de uma
das três categorias abaixo.

\begin{itemize}
\item representações gráficas: árvore e grafo de sintaxe;
\item notações pré e pós fixadas;
\item código de três-endereços, quadras e triplas.
\end{itemize}


\subsection{Árvore e grafo de sintaxe}
\label{sec:org69cae3d}
Um grafo de sintaxe além de incluir as simplificações da árvore de
sintaxe, faz a fatoração das subexpressões comuns, eliminando-as,
conforme a figura abaixo.

\begin{center}
\includegraphics[width=.9\linewidth]{/home/lucas/Pictures/2019-08-24-185929_441x254_scrot.png}
\end{center}


\subsubsection{Funções auxiliares}
\label{sec:org7c412b1}

\textbf{ptr} geraFolha(Categ categ, String lexema) \{\ldots{}\}

\textbf{ptr} geraNo(String opr, ptr node) \{\ldots{}\}

\textbf{ptr} geraNo(String opr, ptr node1, ptr node2) \{\ldots{}\}

Nota que dentro dessas funções se faz necessário verificar se os
nós que estão querendo ser criados já existem, caso eles existam,
basta retornar o ponteiro para esses nós já existentes.

\subsubsection{Exemplo}
\label{sec:org9ca8ea1}
Exemplo de gramática com ações semânticas que geram um grafo de
sintaxe.

S = Ea '=' \{ s.ptr = geraNo(atr.lex, Ea.ptr); \}

Ea = Ea1 'opa' Ta \{ Ea.ptr = geraNo(opa.lex, Ea1.ptr, Ta.ptr); \}

Ea = Ta \{ Ea.ptr = Ta.ptr; \}

Ta = Ta1 'opm' Fa \{ Ta.ptr = geraNo(opm.lex, Ta1.ptr, Fa.ptr); \}

Ta = Fa \{ Ta.ptr = Fa.ptr; \}

Fa = '(' Ea ')' \{ Fa.ptr = Ea.ptr; \}

Fa = 'cten' \{ Fa.ptr = geraFolha('cten', cten.lex); \}

Fa = 'id' \{ Fa.ptr = geraFolha('id', id.lex); \}


\subsection{Notações pré e pós-fixadas}
\label{sec:org38db099}
Notações pré e pós-fixadas podem ser generalizadas para operadores
\textbf{n-ários}. Para a avaliação de expressões desse tipo, pode-se
utilizar uma pilha e um processo que age do seguinte modo: lê a
expressão da esquerda para a direita, empilhando cada operando até
encontrar um operador. Encontrando um operador n-ário, aplica o
operador aos n operandos do topo da pilha. Processamento semelhante
pode ser aplicado para a avaliação de expressões pré-fixadas; nesse
caso, a expressão é lida da direita para a esquerda.

\begin{center}
\includegraphics[width=.9\linewidth]{/home/lucas/Pictures/2019-08-24-191620_444x151_scrot.png}
\end{center}

\subsubsection{Exemplo}
\label{sec:org5d9097a}
Exemplo de gramática com ações semânticas que geram uma notação
pós-fixada.

S = Ea '=' \{ print(Ea.cod); \}

Ea = Ea1 'opa' Ta \{ Ea.cod = Ea1.cod + Ta.cod + opa.lex; \}

Ea = Ta \{ Ea.cod = Ta.cod \}

Ta = Ta1 'opm' Fa \{ Ta.cod = Ta1.cod + Fa.cod + opm.lex; \}

Ta = Fa \{ Ta.cod = Fa.cod; \}

Fa = '(' Ea ')' \{ Fa.cod = Ea.cod; \}

Fa = 'cten' \{ Fa.cod = cten.lex; \}

Fa = 'id' \{ Fa.cod = id.lex; \}

\subsection{Código de 3 endereços e quadras}
\label{sec:org37465e8}
No código intermediário de 3 endereços, cada instrução faz
referência a no máximo três variáveis(endereços de memória).
O tipo de código intermediário que será utilizado na disciplina
é o de quadras.

A principal diferença entre os dois é a ocupação de memória, que
nos dias de hoje não é um problema, devido ao fato de que a maioria
dos dispositivos tem memória o suficiente para suportar quadras.

\subsubsection{Funções auxiliares}
\label{sec:orga44d705}
\textbf{String} geraTemp() \{\ldots{}\}

\textbf{String} geraRot() \{\ldots{}\}

\textbf{void} emiteRot() \{\ldots{}\}

\textbf{String} geraCod(String operador, String operando1, String
operando2, String destino) \{\ldots{}\}

\textbf{String} geraCod(String operador, String operando, String destino)
\{\ldots{}\}

\textbf{String} geraGt(String rotulo) \{\ldots{}\}


\subsubsection{Exemplo}
\label{sec:orgb50a82b}

S = 'id' '=' Ea \{

S.cod = Ea.cod + geraCod('=', Ea.nome, id.lex);

\}

Ea = Ea1 'opa' Ta \{ 

Ea.nome = geraTemp();

Ea.cod = Ea1.cod + Ta.cod + geraCod(opa.lex, Ea1.nome,
Ta.nome,

Ea.nome);

\}

Ea = Ta \{ 

Ea.nome = Ta.nome;

Ea.cod = Ta.cod;

\}

Ta = Ta1 'opm' Fa \{ 

Ta.nome = geraTemp();

Ta.cod = Ta1.code + Fa.cod + geraCod(opm.lex, Ta1.nome,

Fa.nome, Ta.nome);

\}

Ta = Fa \{

Ta.nome = Fa.nome;

Ta.cod = Fa.cod;

\}

Fa = '(' Ea ')' \{

Fa.nome = Ea.nome;

Fa.cod = Ea.cod;

\}

Fa = '-' Fa1 \{

Fa.nome = geraTemp();

Fa.cod = Fa1.cod + geraCod("-u", Fa1.nome, Fa.nome);

\}

Fa = 'cten' \{

Fa.nome = cten.lex;

Fa.cod = "";

\}

Fa = 'id' \{

Fa.nome = id.lex;

Fa.cod = "";

\}


\subsection{Ações semânticas para construção de tabela de símbolos}
\label{sec:orgadfa34d}
Esta seção apresenta esquemas de tradução que reconhecem
declarações de variáveis e geram tabelas de símbolos. Inicialmente,
é apresentado um esquema que gera tabelas de símbolos para
programas monolíticos, isto é, formados por um único
bloco. Posteriormente, esse esquema é estendido para permitir a
geração de tabelas para programas bloco-estruturados com
procedimentos aninhados.


\subsection{Expressões lógicas e comandos de controle}
\label{sec:orgd6bb2b8}
Expressões lógicas são usadas como expressões condicionais em
comandos de controle e em comandos de atribuição lógica. Nesta
seção, apresentaremos dois tipos de traduções para expressões
lógicas.

\begin{itemize}
\item Representação numérica: este método codifica numericamente as
constantes true e false e avalia as expressões lógicas de forma
numérica, ficando o resultado de avaliação numa variável
temporária.
\item Representação por fluxo de controle: este método traduz
expressões lógicas para instruções if e goto que desviam a
execução do programa para pontos distintos, caso o resultado da
avaliação seja true ou false.
\end{itemize}

\subsubsection{Representação numérica}
\label{sec:org1599202}
Esquema de tradução para avaliação numérica de expressões
lógicas. Note que o esquema de tradução abaixo gera código para
expressões lógicas, supondo que as instruções geradas são
armazenadas num vetor de quadruplas. A função geraCod, nesse caso,
utiliza uma variável proxq para indicar o índice da próxima quadra
disponível. Após gravar uma quadrupla, a função geraCod incrementa
proxq.

\textbf{Exemplo:}

Eb = Eb1 'ou' Tb \{

NTerm Eb;

Eb.nome = geraTemp();

Eb.cod = Eb1.cod + Tb.cod + geraCod('or', Eb1.nome, Tb.nome,

Eb.nome);

\}

Eb = Tb \{ NTerm Eb = Tb; \}

Tb = Tb1 'e' Fb \{

NTerm tb;

tb.nome = geraTemp();

tb.cod = Tb1.cod + Fb.cod + geraCod('e', Tb1.nome, Fb.nome,

tb.nome);

\}

Tb = Fb \{ NTerm Tb = Fb; \}

Fb = 'nao' Fb1 \{

NTerm Fb;

Fb.nome = geraTemp();

Fb.cod = Fb1.cod + geraCod('nao', Fb1.cod, Fb.nome);

\}

Fb = '(' Eb ')' \{ NTerm Fb = Eb; \}

Fb = Ea1 'opr' Ea2 \{

NTerm Fb;

Fb.nome = geraTemp();

Fb.cod = Ea1.cod + Ea2.cod + geraCod(opr.lex, Ea1.nome,
Ea2.nome,

geraRot(proxq+3)) + geraCod('=', '0', Fb.nome) +

geraGT(geraRot(proxq+2)) + geraCod('=', '1', Fb.nome);

\}

Fb = 'verd' \{ NTerm fb; fb.nome = geraTemp(); fb.cod =

geraCod('=', '1', Fb.nome); 

\}

Fb = 'falso' \{ NTerm fb; fb.nome = geraTemp(); fb.cod =

geraCod('=', '0', Fb.nome);

\}


Esquema de tradução para gramáticas com while

Sent = 'enquanto' Eb 'faça' LSent 'fim' \{

String inicio = geraRot();

String fim = geraRot();

Sent.cod = emiteRot(inicio) + Eb.cod + geraCod('==', Eb.nome, '0',

fim) + LSent.cod + geraGT(inicio) + emiteRot(fim);

\}

\subsubsection{Representação por fluxo de controle}
\label{sec:org149db06}
Este método traduz expressões lógicas para um código formado por
instruções \textbf{if} e \textbf{goto}. São gerados rótulos que serão atributes
chamados \textbf{true} e \textbf{false} para onde a execução deve ser
transferida em caso de avaliação verdadeira ou falsa.

\textbf{Exemplo:}

Sent = 'enquanto' Eb 'repita' LSent 'fim' \{

NTBool Eb; Eb.true = geraRot(); Eb.false = Sent.prox;

NTSent LSent; LSent.prox = Sent.prox;

Sent.cod = emiteRot(LSent.prox) + Eb.cod + emiteRot(Eb.true) + LSent.cod + geraGT(LSent.prox);

\}

Sent = 'se' Eb 'entao' LSent Senao \{

NTBool Eb; Eb.true = geraRot(); Eb.false = geraRot();

NTSent LSent; Lsent.prox = Sent.prox;

NTSent Senao; Senao.prox = Sent.prox;

Sent.cod = Eb.cod + emiteRot(Eb.true) + LSent.cod +

geraGT(LSent.prox) + emiteRot(Eb.false) + Senao.cod + geraGT(Senao.prox);

\}

Senao = 'senao' LSent 'fim' \{ 

NTSent LSent; LSent.prox = Senao.prox;

Senao.cod = Lsent.cod + geraGT(Lsent.prox);

\}

Senao = 'fim' \{

Senao.cod = "";

\}

Eb = Eb1 'ou' Tb \{

NTBool Eb1; Eb1.true = Eb.true; Eb1.false = geraRot();

NTBool Tb; Tb.true = Eb.true; Tb.false = Eb.false;

Eb.cod = Eb1.cod + emiteRot(Eb1.false) + Tb.cod;

\}

Eb = Tb \{ NTBool Tb = Eb; \}

Tb = Tb1 'e' Fb \{

NTBool Tb1; Tb1.false = Tb.false; Tb1.true = geraRot();

NTBool Fb; Fb.false = Tb.false; Fb.true = Tb.true;

Tb.cod = Tb1.cod + emiteRot(Tb1.true) + Fb.cod;

\}

Tb = Fb \{ NTBool Fb = Tb; \}


Fb = 'nao' Fb1 \{ 

NTBool Fb1; Fb1.true = Fb.false; Fb1.false = Fb.true; 

Fb.cod = Fb1.cod;

\}

Fb = Ea1 'opr' Ea2 \{

NTerm Ea1, Ea2;

Fb.cod = Ea1.cod + Ea2.cod + geraCod(opr.lex, Ea1.nome,

Ea2.nome, Fb.true) + geraGT(Fb.false);

\}

Fb = 'true' \{ Fb.cod = geraGT(Fb.true); \}

Fb = 'false' \{ Fb.cod = geraGT(Fb.falso); \}
\end{document}
