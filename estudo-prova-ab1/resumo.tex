% Created 2019-07-20 sáb 00:43
% Intended LaTeX compiler: pdflatex
\documentclass[11pt]{article}
\usepackage[utf8]{inputenc}
\usepackage[T1]{fontenc}
\usepackage{graphicx}
\usepackage{grffile}
\usepackage{longtable}
\usepackage{wrapfig}
\usepackage{rotating}
\usepackage[normalem]{ulem}
\usepackage{amsmath}
\usepackage{textcomp}
\usepackage{amssymb}
\usepackage{capt-of}
\usepackage{hyperref}
\author{Lucas Peixoto}
\date{\today}
\title{}
\hypersetup{
 pdfauthor={Lucas Peixoto},
 pdftitle={},
 pdfkeywords={},
 pdfsubject={},
 pdfcreator={Emacs 26.2 (Org mode 9.1.9)}, 
 pdflang={English}}
\begin{document}

\tableofcontents

Resumo de compiladores

\section{Interpretação Pura}
\label{sec:orgc448c15}
Uma interpretação pura em compiladores é feita da seguinte maneira:
dado o programa fonte, o programa fonte passará por um
interpretador. Esse \textbf{interpretador} irá interpretar o código fonte a
partir de dados de entrada e bibliotecas, gerando assim os
resuldados.

\subsection{Benefícios}
\label{sec:org6c057c9}
\begin{itemize}
\item Os erros são mais fácies de serem detectados
\item Além de testar pedaços de códigos
\end{itemize}

\subsection{Desvantagens}
\label{sec:org023f03b}
\begin{itemize}
\item A execução é mais lenta
\end{itemize}

\section{Compilação para bytecode}
\label{sec:org954d377}
O uso de compilação para geração de bytecode faz com que o programa
fonte passe pelas seguintes partes:

\subsection{Analisador léxico}
\label{sec:org650e104}
O analisador léxico verifica se os elementos presentes no código
fonte são válidos na linguagem. O seu papel é segmentar o programa
e verificar a validade de cada segmento de código. A saída do
analisador léxico são tokens. Tokens possuem um lexema, que nada
mais é do que a classificação do token, linha, coluna e valor do
token, que nada mais é do que o conteúdo em si.

\subsection{Analisador sintático}
\label{sec:org963d3a3}
Tem como saída uma árvore de derivações. Esse analisador tem como
objetivo requisitar tokens do analisador léxico e a partir desses
tokens fazer uma análise sintática do programa fonte. A medida que
a análise sintática é feita, o analisador vai montando a árvore de
derivação. Existem vários tipos de analisadores sintáticos.

\subsection{Analisador semântico}
\label{sec:org357c967}
A partir da ávore de derivação criada pelo analisador sintático, o
analisador semântico avalia a compatibilidade dos nós e folhas
dessa árvore. Ou seja, é no analisador semântico que a verificação
dos tipos é efetuada, vendo assim se o programa fonte faz sentido
segundo as especificações da linguagem.

\subsection{Gerador de bytecode}
\label{sec:org457c11e}
Esse processo não é coberto pela disciplina.

\subsection{Tabela de símbolos}
\label{sec:orgae38d12}
Estabelece uma comunicação entre fases(citadas acima) diferentes do
compilador. Geralmente na tabela de símbolos temos informações a
respeito de variávies, funções/procedimentos e parâmetros.
\section{Análise Léxica(Scanner)}
\label{sec:orgbd0e223}
Lê o código fonte e o separa em lexemas(unidades
léxicas). Identifica os lexemas, categorizando-os e guardando suas
posições. Elimina comentários.

Lexemas serão palavras reservadas ou não. De modo geral os lexemas
podem ser :
\begin{itemize}
\item Identificadores
\item Constantes
\item Operadores
\item Separadores
\item Palavras reservadas
\end{itemize}

Para fazer a identificação de cada um desses tipos, pode-se usar:
\begin{itemize}
\item Expressões regulares
\item Gramática regular
\item Autômato finito determinístico
\item Autômato finito não-determinístico
\end{itemize}

O alfabeto léxico se trata dos caracteres permitidos no código fonte,
enquanto o alfabeto sintático se refere aos tokens possíveis.

\section{Análise sintática(Parser)}
\label{sec:orgfe88b1a}
É a segunda fase de um compilador. Atua sobre uma GLC, dado o
programa fonte e seus respectivos tokens, produz-se uma árvore de
derivação que reflete a estrutura sintática da sentença.

Existem algumas estratégias de análise:
\begin{itemize}
\item Top-down  ou descendente
\begin{itemize}
\item Backtracking
\item Preditiva recursiva
\item Preditiva tabular
\end{itemize}
\item Bottom-up ou ascendente(redutiva, empilha-reduz)
\begin{itemize}
\item Precedência de operadores
\item SLR(Simple left to right rightmost)
\end{itemize}
\end{itemize}


\section{Análise Top-Down}
\label{sec:org64cfc39}
A análise se inicia pela raiz da árvore através de
derivação. Produções ao invés de erivações são escolhidas com base
na entrada.

\subsection{Análise com Backtracking}
\label{sec:orgd311655}
A cada derivação com mais de uma opção, cria-se um checkpoint que
permite retorno em caso de erro

\subsubsection{Desvantagens}
\label{sec:org9153f67}
\begin{itemize}
\item É difícil encontrar a causa do erro
\item Ineficiente
\end{itemize}
\subsubsection{Vantagens}
\label{sec:orgdef5e7d}
\begin{itemize}
\item É capaz de analisar qualquer tipo de gramática tratável
\end{itemize}

\subsection{Análise recursiva preditiva}
\label{sec:org7674257}
A gramática não pode ter recursões à esquerda. Logo ela deve ser:
\begin{itemize}
\item Fatorada
\item Primeiros terminais deriváveis de produções de um mesmo
não-terminal não podem ter intersecções
\end{itemize}

\subsubsection{Características}
\label{sec:org7ca3028}
\begin{itemize}
\item Fácil de implementar
\item Muito código
\item Permite tratar repetições e opcionalidades
\end{itemize}

Nesse tipo de analisador, o código é tratado sintáticamente da
esquerda para a direita, onde para cada símbolo não terminal existe
uma função que trata esse símbolo.

\subsection{Análise preditiva tabular(não recursiva)}
\label{sec:org316093c}
Para esse tipo de análise a gramática também deve ser fatorada e os
primeiros temrinais deriváveis de um mesmo não-terminal não podem
ter intersecções.
\subsubsection{Funcionamento}
\label{sec:org4b56726}
\begin{enumerate}
\item Inicia-se o processamento lendo o primeiro token e com o
símbolo inicial na pilha.
\item Se entrada for EOF e a pilha estiver vazia a sentença é aceita.
\item Se token na entrada corresponder ao terminal no topo da pilha,
o token foi reconhecido, acessa-se o próximo token e
desempilha-se o terminal
\item Topo da pilha não-terminal, indexa-se a tabela de análise pelo
mesmo e pelo token na entrada, se a casa indexada estiver vazia
é um erro sintático, caso contrário, efetua-se a derivação indicada.
\begin{enumerate}
\item Desempilha-se o não-terminal e empilha o lado direito da
produção na ordem inversa.
\end{enumerate}
\end{enumerate}


\subsubsection{Exemplo:}
\label{sec:orga4f6518}
\begin{itemize}
\item Gramática\\
(1) Eb = Tb Ebr\\
(2) Ebr = 'ou' Tb Ebr\\
(3) Ebr = Vazio\\
(4) Tb = Fb Tbr\\
(5) Tbr = 'e' Fb Tbr\\
(6) Tbr = Vazio\\
(7) Fb = 'não' Fb\\
(8) Fb = '(' Eb ')'\\
(9) Fb = 'id'\\
(10) Fb = 'verd'\\
(11) Fb = 'falso'\\

\item Tabela de análise com recuperação de erro local\\

Apenas uma modificação é feita:
\begin{itemize}
\item Desempilha um símbolo
\item Empilha um símbolo
\item Introduz um terminal na entrada
\item Trocar terminal da entrada\\
\end{itemize}

Tabela:
\begin{itemize}
\item é adicionada uma linha para os terminais que apareçam 
fora da 1° posição em algum handle.
\item Expande as produções vazias para posições em branco na mesma
linha (porterga a identificação de erro)\\
\end{itemize}

Tipos de erros encontrados no exemplo:
\begin{itemize}
\item Err1:
\begin{itemize}
\item Msg: "Operando esperado"
\item Ação: "Insere "id" na entrada"
\end{itemize}
\item Err2:
\begin{itemize}
\item Msg: " ')' esperado"
\item Ação: Desempilha ')'
\end{itemize}
\item Err3:
\begin{itemize}
\item Msg: "EOF esperado"
\item Ação: Elimina a lista restante da entrada
\end{itemize}
\end{itemize}
\end{itemize}

\begin{center}
\begin{tabular}{l|lllllllll|}
\hline
 & 'ou' & 'e' & 'nao' & '(' & ')' & 'id' & 'verd' & 'falso' & EOF\\
\hline
\hline
Eb & Err1 & Err2 & d1 & d1 & Err1 & d1 & d1 & d1 & Err1\\
\hline
Ebr & d2 & Vazio & Vazio & Vazio & Vazio-d3 & Vazio & vazio & Vazaio & Vazio-d3\\
\hline
Tb & Err1 & Err1 & d4 & d4 & Err1 & d4 & d4 & d4 & Err1\\
\hline
Tbr & Vazio-d6 & d5 & Vazio & Vazio & Vazio-d6 & Vazio & Vazio & Vazio & Vazio-d6\\
\hline
Fb & Err1 & Err1 & d7 & d8 & Err1 & d9 & d10 & d11 & Err1\\
\hline
')' & Err2 & Err2 & Err2 & Err2 & DAT & Err2 & Err2 & Err2 & Err2\\
\hline
PV & Err3 & Err3 & Err3 & Err3 & Err3 & Err3 & Err3 & Err3 & AC\\
\hline
\end{tabular}
\end{center}


\begin{itemize}
\item Tabela de análise com recuperação de erro pânico\\
É necessário identificar os tokens de sincronização. Para cada
não terminal temos que o token sync desse não terminal é dado
pelo follow dele.

Tratamento:
\begin{itemize}
\item Mensagem de erro
\item EOF na entrada: encerra a análise
\item Pilha:
\begin{itemize}
\item Terminal:
\begin{itemize}
\item Despreza tokens na entrada enquanto terminal da pilha
for diferente do token de entrada
\item Se terminal corresponder ao token de entrada continua a
análise
\end{itemize}
\item Não-terminal
\begin{itemize}
\item Despreza tokens na entrada enquanto entrada não
corresponder a um terminal de sincronização
\item Quando corresponder: desempilha o não terminal e continua
a análise
\end{itemize}
\end{itemize}
\end{itemize}
\end{itemize}

\begin{center}
\begin{tabular}{l|lllllllll|}
\hline
 & 'ou' & 'e' & 'nao' & '(' & ')' & 'id' & 'verd' & 'falso' & EOF\\
\hline
\hline
Eb & - & - & d1 & d1 & Sync & d1 & d1 & d1 & Sync\\
\hline
Ebr & d2 & - & - & - & Vazio-d3 & - & - & - & Vazio-d3\\
\hline
Tb & - & - & d4 & d4 & Sync & d4 & d4 & d4 & Sync\\
\hline
Tbr & Vazio-d6 & d5 & - & - & Vazio-d6 & - & - & - & Vazio-d6\\
\hline
Fb & Sync & Sync & d7 & d8 & Sync & d9 & d10 & d11 & Sync\\
\hline
\end{tabular}
\end{center}

\section{Análise bottom-up}
\label{sec:org018276e}
\begin{itemize}
\item Redutiva
\item Empilha/reduz
\item A partir das folhas da árvore
\item Inicia com a pilha vazia
\item Empilha símbolos até que na pilha estejam símbolos que
correspondam ao lado direito de uma produção (\textbf{handle})
\item Este \textbf{handle} é desempilhado e em seu lugar é empilhado o símbolo
do lado esquerdo da produção(redução)
\item São empilhados tokens ou símbolos não terminais quando ocorre uma
redução
\item A produção a ser usada só é conhecida no momento da redução
\item Aceita quando na pilher tiver o símbolo inicial da GLC e na
entrada EOF
\item Se a gramática for não ambigua, então toda forma sentencial gerada
por G tem exatamente um handle.
\end{itemize}
\subsection{Analisadores}
\label{sec:org8db703a}
\begin{itemize}
\item Ações:
\begin{itemize}
\item Empilha (EAT): empilha o token na entrada e acessa o próximo
token
\item Reduz(ri): substitui o handle da produção (i) pelo terminal à
esquerda nesta produção
\item Aceita(Ac): reconhece a sentença quando na pilha tem o símbolo
inicial e EOF na entrada
\item Erro: demais casos
\end{itemize}
\item Tipos de analizadores:
\begin{itemize}
\item Precedência de operadores: bom para reconhecimento de
expressões
\item LR:
\begin{itemize}
\item SLR: Simple LR (só veremos esse)
\item LALR: Look Ahead LR
\item LR canônico
\end{itemize}
\end{itemize}
\end{itemize}
\subsection{Analisador de precedência de operadores}
\label{sec:orgb286ed0}
\begin{itemize}
\item Gramática de operadores
\item Gramática simplificada
\item Não pode ter dois não terminais adjascentes
\item Não podem ter produções vazias
\item Não ode ter um operador com aridades diferentes
\item A gramática não tem informações de rpecedência e associatividade
\begin{itemize}
\item esta informação deve ser colocada na tabela de análise
\end{itemize}
\item Tabela de análise:
\begin{itemize}
\item Tokens na pilha x tokens na entrada, PST(pilha sem token)
\item entrada(colunas): tokens e EOF
\item PST pode estar vazia (PV) ou não ter um não terminal
\end{itemize}
\end{itemize}
\subsubsection{Relação entre tokens + ao topo topo/PST e a entrada(tokens ou EOF)}
\label{sec:org97c426e}
\begin{itemize}
\item Prioridade calculada em relação da precedência e da
associatividade
\item Se token na pilha tem prioridade:
\begin{itemize}
\item maior: reduz
\item menor: empilha entrada
\item igual: empilha entrada
\end{itemize}
\item APO não garante validade do handle
\item Na redução os elementos da pilha devem ser testados para
garantir esta validade
\begin{itemize}
\item Caso isso não ocorra, temos um erro na redução chamado de erro
mais tarde
\end{itemize}
\item Erros identificados durante a análise, especificados na tabela
de análise, são ditos erros mais cedo
\end{itemize}
\subsubsection{Exemplo}
\label{sec:org28e5308}
\begin{itemize}
\item Gramática\\
(1) Eb = Eb 'ou' Eb\\
(2) ---- Eb 'e' Eb\\
(3) ---- 'nao' Eb\\
(4) ---- '(' Eb ')'\\
(5) ---- 'id'\\
(6) ---- 'verd'\\
(7) ---- 'falso\\
\end{itemize}

Tipos de erro mais cedo encontrados:
\begin{itemize}
\item Err1:
msg: ')' esperado
ação: encerra análise
\item Err2:
msg: operador binário esperado
ação: insere um operador binário
\item Err3: ')' não esperado
ação: remove ')' da entrada
\end{itemize}

Erros mais tarde que podem ocorrer:
\begin{itemize}
\item (1) e (2):\\
Se topo != Eb\\
  msg: Segundo operando faltando\\
senão desempilha Eb\\
desempilha 'ou'/'e'\\
se topo != Eb\\
  msg: Primeiro operando faltando\\
senão desempilha Eb\\
empilha novo Eb\\
\item (3): \\
se topo != Eb
msg: operando faltando
senão desempilha Eb
desempilha 'nao'
empilha novo Eb
\item (4): \\
desempilha ')'
se topo != Eb
msg: expressão booleana faltando
senão desempilha Eb
desempilha '('
empilha novo Eb
\item (5), (6) e (7):
não existe o erro porque nunca é empilhado nada sobre esses
terminais.
\end{itemize}


\begin{center}
\begin{tabular}{l|lllllllll|}
\hline
 & 'ou' & 'e' & 'nao' & '(' & ')' & 'id' & 'verd' & 'falso' & EOF\\
\hline
\hline
'ou' & r1 & EAT & EAT & EAT & r1 & EAT & EAT & EAT & r1\\
\hline
'e' & r2 & r2 & EAT & EAT & r2 & EAT & EAT & EAT & r2\\
\hline
'nao' & r3 & r3 & EAT & EAT & r3 & EAT & EAT & EAT & r3\\
\hline
'(' & EAT & EAT & EAT & EAT & EAT & EAT & EAT & EAT & Err1\\
\hline
')' & r4 & r4 & Err2 & Err2 & r4 & Err2 & Err2 & Err2 & r4\\
\hline
'id' & r5 & r5 & Err2 & Err2 & r5 & Err2 & Err2 & Err2 & r5\\
\hline
'verd' & r6 & r6 & Err2 & Err2 & r6 & Err2 & Err2 & Err2 & r6\\
\hline
'falso' & r7 & r7 & Err2 & Err2 & r7 & Err2 & Err2 & Err2 & r7\\
\hline
PST & EAT & EAT & EAT & EAT & Err3 & EAT & EAT & EAT & AC\\
\hline
\end{tabular}
\end{center}

\subsection{Analisador SLR}
\label{sec:orgbd89409}
Conjunto canônico de itens SLR
\begin{itemize}
\item Item Lr(0) em uma GLC marca uma posição no andamendo da análise
marcada por um ponto.
\end{itemize}

\subsubsection{Exemplo}
\label{sec:org92cf842}
\begin{itemize}
\item Gramática\\
(0) S = Eb\\
(1) Eb = Eb 'ou' Tb\\
(2) Eb = Tb\\
(3) Tb = Tb 'e' Fb\\
(4) Tb = Fb\\
(5) Fb = 'nao' Fb\\
(6) Fb = '(' Eb ')'\\
(7) Fb = 'id'\\
\end{itemize}

Fazendo o fechamento da gramática
\begin{itemize}
\item 0 = \{S = \textbf{.} Eb, Eb = \textbf{.} Eb 'ou' Tb, Eb = \textbf{.} Tb, Tb = \textbf{.} Tb
'e' Fb, Tb = \textbf{.} Fb, Fb = \textbf{.} 'nao' Fb, Fb = \textbf{.} '(' Eb ')', Fb
= \textbf{.} 'id'\}
\item 1 = (0, Eb) = \{S = Eb \textbf{.}, Eb = Eb \textbf{.} 'ou' Tb\}
\item 2 = (0, Tb) = \{Eb = Tb \textbf{.}, Tb = Tb \textbf{.} 'e' Fb\}
\item 3 = (0, Fb) = \{Tb = Fb \textbf{.}\}
\item 4 = (0, 'nao') = \{ Fb = 'nao' \textbf{.} Fb, Fb = \textbf{.} 'nao' Fb, Fb =
\textbf{.} '(' Eb ')', Fb = \textbf{.} 'id'\}
\item 5 = (0, '(') = \{ Fb = '(' \textbf{.} Eb ')', Eb = \textbf{.} Eb 'ou' Tb, 
Eb = \textbf{.} Tb, Tb = \textbf{.} Tb 'e' Fb, Tb = \textbf{.} Fb, 
Fb = \textbf{.} 'nao' Fb, Fb = \textbf{.} '(' Eb ')', Fb = \textbf{.} 'id'\}
\item 6 = (0, 'id') = \{ Fb = 'id' \textbf{.}\}
\item 7 = (1, 'ou') = \{ Eb = Eb 'ou' \textbf{.} Tb, Tb = \textbf{.} Tb
'e' Fb, Tb = \textbf{.} Fb, Fb = \textbf{.} 'nao' Fb, Fb = \textbf{.} '(' Eb ')', Fb
= \textbf{.} 'id'\}
\item 8 = (2, 'e') = \{ Tb = Tb 'e' \textbf{.} Fb, Fb = \textbf{.} 'nao' Fb,
Fb = \textbf{.} '(' Eb ')', Fb = \textbf{.} 'id'\}
\item 9 = (4, Fb) = \{Fb = 'nao' Fb \textbf{.}\}
\item (4, 'nao') = 4
\item (4, '(') = 5
\item (4, 'id') = 6
\item 10 = (5, Eb) = \{Fb = '(' Eb \textbf{.} ')', Eb = Eb \textbf{.} 'ou' Tb\}
\item (5, Tb) = 2
\item (5, Fb) = 3
\item (5, 'nao') = 4
\item (5,  '(') = 5
\item (5, 'id') = 6
\item 11 = (7, Tb) = \{ Eb = Eb 'ou' Tb \textbf{.}, Tb = Tb \textbf{.} 'e' Fb \}
\item (7, Fb) = 3
\item (7, '(') = 5
\item (7, 'nao') = 4
\item (7, 'id') = 6
\item 12 = (8, Fb) = \{ Tb = Tb 'e' Fb \textbf{.}\}
\item (8, 'nao') = 4
\item (8, '(') = 5
\item (8, 'id') = 6
\item 13 = (10, ')') = \{ Fb = '(' Eb ')' \textbf{.}\}
\item (10, 'ou') = 7
\item (11, 'e') = 8
\end{itemize}
\end{document}